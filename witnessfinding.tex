%% witnessfinding.tex - Lower bounds for witness-finding algorithms
%%
%% Copyright 2014 Jeffrey Finkelstein.
%%
%% This LaTeX markup document is made available under the terms of the Creative
%% Commons Attribution-ShareAlike 4.0 International License,
%% https://creativecommons.org/licenses/by-sa/4.0/.
\documentclass{article}

%% This must come before hyperref.
\usepackage{amsthm}
%% This must come before hyperref.
\usepackage{thmtools}
%% This must come before complexity.
\usepackage{hyperref}
%% This is strongly recommended by biblatex.
\usepackage[english]{babel}
%% This is strongly recommended by biblatex.
\usepackage{csquotes}
\usepackage[backend=biber]{biblatex}
\usepackage{amsmath}
%\usepackage{amssymb}
\usepackage{complexity}

\hypersetup{pdftitle={Lower bounds for witness-finding algorithms}, pdfauthor={Jeffrey Finkelstein}}

\addbibresource{witnessfinding.bib}

\declaretheorem[numberwithin=section]{theorem}
\declaretheorem[numberlike=theorem]{lemma}
\declaretheorem[numberlike=theorem]{corollary}
\declaretheorem[numberlike=theorem, style=definition]{definition}

\newcommand{\pow}{\mathcal{P}}
\newcommand{\todo}[1]{\textbf{TODO #1}}

\title{Lower bounds for witness-finding algorithms}
\author{Jeffrey Finkelstein}

\begin{document}

\maketitle

%%\section{Introduction}
%%
%% context
%% need
%% task
%% object
%%
%% findings
%% conclusion
%% perspectives

\section{Definitions}

Let $\Sigma = \{0, 1\}$.
Let $[n] = \{1, \dotsc, n\}$.

A language $L \subseteq \Sigma^*$ is a \emph{witnessed language} if there exists a language $L' \subseteq \Sigma^* \times \Sigma^*$, called the \emph{witness relation}, such that $x \in L$ if and only if there is a $w \in \Sigma^*$ such that $(x, w) \in L'$.

\begin{definition}[{\autocite[Definition~1]{krw12}}]
  Suppose $n$ is a positive integer and $k \in [2^n]$.
  A \emph{witness set} is a nonempty subset of $\Sigma^n$ denoted by $W$, and each element of $W$ is a \emph{witness}.
  An \emph{intersection query} (contrast with intersection queries defined in \autocite{krw14}) is a set $Q \subseteq \Sigma^n$ interpreted as the question ``does $|Q \cap W| = k$?''.

  An \emph{intersection oracle} is a function $A^k_W \colon \Sigma^n \to \{0, 1\}$ such that $A^k_W(Q) = 1$ if and only if $|Q \cap W| = k$.
  %% Let $A_W(Q) \in \{0, 1\}$ denote the answer to the query $Q$, formally $A_W(Q) = 1$ if and only if $Q \cap W \noteq \emptyset$.
\end{definition}

If $k = 0$, the intersection query is called an \emph{emptiness query} and the intersection oracle $A^0_W$ is called the \emph{emptiness oracle}.
If $k = 1$, the intersection query is called an \emph{isolation query} and the intersection oracle $A^0_W$ is called the \emph{isolation oracle}.

Suppose a language $L$ is a witnessed language with witness relation $L'$.
If $L'$ is in $\P$, then the emptiness oracle can be interpreted as a $\coNP$ oracle (which is equivalent to an $\NP$ oracle by complementation) and the isolation oracle can be interpreted as a $\US$ oracle ($\US$ was defined in \autocite{bg82}).
In other words, $A^0_W$ acts as an oracle for \textsc{Satisfiability} and $A^1_W$ for \textsc{Unique Satisfiability}.
(For context, $\coNP \subseteq \US \subseteq \BH_2 \subseteq \P^\NP$.)

\begin{definition}
  Let $k$ be an integer.
  Suppose, for any fixed witness set $W$, $A^k_W$ is an intersection query oracle.
  A \emph{randomized witness-finding query algorithm} (or more briefly, a witness-finding algorithm) is a pair $(Q, F)$ of randomized Turing machines where, for every witness length $n$ and random seed $s \in \Sigma^r$, $Q(s)$ outputs a sequence of queries $Q_1, \dotsc, Q_m$, each a subset of $\Sigma^n$, and $F(s, A^k_W(Q_1), \dotsc, A^k_W(Q_m))$ outputs an element of $\Sigma^n$.
  For a witness set $W$, the algorithm \emph{succeeds with respect to $W$} if $F$ outputs an element of $W$.
  For each collection $\mathcal{W}$ of nonempty subsets of $\Sigma^n$, the \emph{success probability} of the algorithm is
  \begin{equation*}
    \min_{W \in \mathcal{W}} \Pr_{s \in \Sigma^r} [(Q, F) \text{ succeeds with respect to } W].
  \end{equation*}
  Parameters $r$ and $m$ are called the \emph{seed length} and \emph{query complexity}, respectively.

  If $Q$ has access to the intersection query oracle, then the algorithm is called \emph{adaptive}, otherwise it is called \emph{nonadaptive}.
  An adaptive algorithm can only make queries on $Q_1, \dotsc, Q_{i - 1}$ to generate $Q_i$.
\end{definition}

If not otherwise specified, $\mathcal{W}$ is the power set of $\Sigma^n$, excluding the empty set.

\begin{definition}
  If the oracle $A$ in the above definition is the emptiness oracle, the witness-finding algorithm is called an \emph{emptiness query algorithm}.
  If it is the isolation oracle, the algorithm is called an \emph{isolation query algorithm}.
\end{definition}

The emptiness query algorithm is the same as \autocite[Definition~2]{krw12}.
I don't believe the isolation query algorithm has been defined before.

\begin{definition}
  A \emph{randomized hitting algorithm} (or more briefly, a hitting algorithm) is a randomized Turing machine $P$ with access to the isolation oracle $A^1_W$ where, for every witness length $n$ and random seed $s \in \Sigma^r$, $P(s)$ outputs a sequence of queries $Q_1, \dotsc, Q_m$, each a subset of $\Sigma^n$.
  Each $Q_i$ may depend on the oracle answers to $Q_1, \dotsc, Q_{i - 1}$.
  For a witness set $W$, the algorithm \emph{succeeds with respect to $W$} if $A^1_W(Q_m) = 1$.
  For each collection $\mathcal{W}$ of nonempty subsets of $\Sigma^n$, the \emph{success probability} of the algorithm is
  \begin{equation*}
    \min_{W \in \mathcal{W}} \Pr_{s \in \Sigma^r} [P \text{ succeeds with respect to } W].
  \end{equation*}
  Parameters $r$ and $m$ are called the \emph{seed length} and \emph{query complexity}, respectively.
\end{definition}

This definition is a complexity theoretic restatement of the combinatorial ``hitting game'', as described in \autocite[Section~3]{newport14} (the original definition seems to be from \autocite[Definition~5]{bgi92}).

A hitting algorithm is always an adaptive algorithm; a nonadaptive hitting algorithm makes little sense here because it would be essentially the same as simply guessing the witness set at random.

A witness-finding algorithm must output an element of $W$, whereas a hitting algorithm need not.

\todo{define polynomial-time algorithms (provide polysize circuits that encode the query sets)}

\section{Relationship among algorithms}

The problem of finding a hitting algorithm reduces to the problem of finding an isolation query algorithm.

\begin{lemma}\label{lem:reduction}
  Suppose there is a (nonadaptive or adaptive) isolation query algorithm with success probability $p$ and query complexity $m$.
  Then there is a hitting algorithm with success probability $p$ and query complexity $m + 1$.
\end{lemma}
\begin{proof}
  Suppose $(Q, F)$ is the isolation query algorithm.
  Construct hitting algorithm $P$ with access to isolation oracle $A^1_W$ as follows.
  \begin{itemize}
  \item simulate $Q(s)$ to generate queries $Q_1, \dotsc, Q_m$ (simulating access to the isolation oracle if adaptivity is required)
  \item send those queries to the oracle $A^1_W$ to get answers $A_1, \dotsc, A_m$
  \item simulate $F(s, A_1, \dotsc, A_m)$ to output a string $w$.
  \item let $Q_{m + 1} = \{ w \}$ and output $Q_1, \dotsc, Q_{m + 1}$
  \end{itemize}

  Let $W$ be an arbitrary witness set and suppose the success probability of the isolation query algorithm $(Q, F)$ is $p$.
  This is the probability that $w \in W$, or in other words, the probability that $|Q_{m + 1} \cap W| = 1$.
  Since $Q_{m + 1}$ is the last query set output by $P$, the probability that $P$ succeeds with respect to $W$ equals the probability that $(Q, F)$ succeeds with respect to $W$.
  Thus $P$ has success probability $p$ and query complexity $m + 1$.
\end{proof}

On the other hand, there does not seem to be a general reduction from an isolation query algorithm to a hitting algorithm.
In order to output a single witness knowing only that a witness is contained in a particular query set doesn't help, unless the query set is small; there is no such guarantee on the size of that set.

\todo{Provide evidence that it is unlikely that the existence of hitting algorithm implies the existence of an isolation query algorithm (for example, by showing that such a premise implies a collapse of the polynomial time hierarchy).}

\section{Upper bounds}

\begin{theorem}\label{thm:naive}
  There is a polynomial time adaptive emptiness query algorithm with success probability $1$ and query complexity $O(n)$.
\end{theorem}
\begin{proof}[Proof idea]
  This is the well-known bit-by-bit $\P^\NP$ computable function for computing a witness for a given string.

  For each $i \in [n]$ and each $b \in \{0, 1\}$, define the query $Q_{i, b}$ inductively (or in other words, adaptively) as follows.
  If $i = 1$, then $Q_{i, b} = \{ b v | v \in {\{0, 1\}}^{n - 1}\}$.
  For each subsequent pair of queries, let $u = b_1 \dotsb b_{i - 1}$, where $b_j$ is a bit that caused the oracle to answer ``yes'' on query $Q_{j, b_j}$ for each $j \in [i - 1]$, and let $Q_{i, b} = \{ u b v | v \in {\{0, 1\}}^{n - i}\}$.
  Finally, given the responses to $Q_{n, 0}$ and $Q_{n, 1}$, which are singleton sets, the algorithm simply outputs the sole element of the query set that caused the oracle to answer ``yes''.

  The total number of queries is $2n$, which is in $O(n)$.
\end{proof}

\begin{theorem}[{\autocite[Proposition~1]{krw12} (citing \autocite{bcgl92})}]
  There is a polynomial time nonadaptive emptiness query algorithm with success probability $\Omega(1)$ and query complexity $O(n^2)$.
\end{theorem}
\begin{proof}[Proof idea]
  Use the Valiant--Vazirani Isolation Lemma in each query to guess each bit of a possible witness (in parallel).

  First, consider a simpler set of queries.
  Let $s$ be a string in $\Sigma^r$.
  For each $i \in [n]$ and each $b \in \Sigma$, let $Q_{i, b} = \{ w \, | \, w \in C_s \land w_i = b\}$, where $C$ is the isolating set from the Valiant--Vazirani Isolation Lemma (for example, the set of all strings whose image is zero under a pairwise independent hash function).
  The Valiant--Vazirani Isolation Lemma proves that this algorithm has success probability $\Omega(\frac{1}{n})$ and query complexity $O(n)$.
  In order to get success probability $\Omega(1)$, we create $O(n)$ collections of $O(n)$ queries, with an independent set $C_{s_k}$ for each collection.
  Now our queries are $Q_{k, i, b} = \{ w \, | \, w \in C_{s_k} \land w_i = b\}$, for each $k \in \left[O(n)\right]$, each $i$, and each $b$.

  (Remember, the oracle answers queries of the form ``is $Q \cap W$ empty?'', so if we want to determine whether $Q \cap W$ is \emph{not} empty, we must complement each oracle's answers.)
\end{proof}

In \autocite{krw12}, the authors call the above theorem the ``truth-table reduction'' analog of the ``many-one reduction'' result of \autocite[Theorem~4.2]{dkvmw13}.
However, this is not strictly precise, since the former involves an emptiness oracle, whereas the latter involves an isolation oracle.
Fortunately, the proof uses the Valiant-Vazirani Isolation Lemma (that is, the sets $C_{s_k}$), so the probability that $|Q \cap W| \neq 0$ equals the probability that $|Q \cap W| = 1$.

\begin{corollary}
  There is a polynomial time nonadaptive isolation query algorithm with success probability $\Omega(1)$ and query complexity $O(n^2)$.
\end{corollary}

\todo{Double check this logic.}

\begin{corollary}
  There is a polynomial time hitting algorithm with success probability $\Omega(1)$ and query complexity $O(n^2)$.
\end{corollary}
\begin{proof}
  Combine the previous corollary with \autoref{lem:reduction}.
\end{proof}

\section{Lower bounds}

\begin{theorem}[{\autocite[Theorem~2]{krw12}}]\label{thm:nonadaptiveemptiness}
  There is no nonadaptive emptiness query algorithm with success probability $\Omega(1)$ and query complexity $o(n^2)$.
\end{theorem}

The proof of this theorem relies on the existence of a distribution on witness sets that places all witnesses far apart, so it is hard to guess both the number and location of the witnesses.
This distribution comes from \autocite[Theorem~4.2]{dkvmw13}.

\begin{theorem}[{\autocite[Theorems~3.3 and 3.4]{newport14}}]
  There is no hitting algorithm with success probability $\Omega(1 - 2^{-n})$ and query complexity $o(n^2)$.
  Furthermore, these bounds hold even when restricted to witness sets of cardinality two.
\end{theorem}

The proof of this theorem relies on the probabilistic method to show the existence of a witness set that is hard to ``hit'', found in \autocite{ablp91}.

\begin{corollary}\label{cor:isolationalg}
  There is no (nonadaptive or adaptive) isolation query algorithm with success probability $\Omega(1 - 2^{-n})$ and query complexity $o(n^2)$.
  Furthermore, these bounds hold even when restricted to witness sets of cardinality two.
\end{corollary}
\begin{proof}
  This is a proof by contradiction.
  If there were such an algorithm, then \autoref{lem:reduction} would imply a hitting algorithm with the same success probability.
  This would contradict the previous theorem, hence no such algorithm exists.
\end{proof}

This corollary is similar to \autoref{thm:nonadaptiveemptiness}, but provides an even stronger lower bound.
Not only does it exclude the possibility of both adaptive and nonadaptive algorithms, it also allows them to have even better success probabilities.
Furthermore, the isolation query oracle (which is like a $\US$ oracle) is a more powerful than the emptiness query oracle (which is like an $\NP$ oracle).

Other types of queries produce the same lower bounds.
Queries of the form ``does $|Q \cap W| = |Q|$?'', or equivalently, ``is $Q \subseteq W$?'', yield the same lower bounds as in \autoref{thm:nonadaptiveemptiness}.
Such queries are \emph{monotone queries} (that is, if $W \subseteq W'$, then $Q \subseteq W$ implies $Q \subseteq W'$), and monotone queries with constant success probability have $o(n^2)$ query complexity \autocite[Theorem~1.3]{krw14}.

\begin{theorem}[{\autocite[Theorem~3.2]{newport14}}]
  There is no hitting algorithm with (expected success probability ??? and) expected query complexity $o(n)$.
\end{theorem}

\todo{I can't understand what the random variable is for which there is an expectation here.}

\section{Todo}

Does the lower bound of \autoref{cor:isolationalg} hold for any algorithm asking queries to an oracle that answers questions of the form ``does $|Q \cap W| = k$?'' for any fixed positive integer $k$?
(It certainly doesn't hold when $k = 0$, because of \autoref{thm:naive}.)
This question may be answered by adapting the proof by probabilistic method given in \autocite{ablp91}.
In \autocite[Remark~2]{krw12}, the authors state without proof that the lower bound in \autoref{thm:nonadaptiveemptiness} holds even for queries of the form ``is $|Q \cap W| > k$?'' (remember, they were originally considering not emptiness queries but \emph{nonemptiness} queries).

\printbibliography{}

\end{document}
